\section{Presentazione}
    Per la realizzazione del sito abbiamo utilizzato il linguaggio CSS utilizzando quasi esclusivamente misure  relative come em e percentuali, cercando di ridurre al minimo l'utilizzo di misure statiche come i px per creare un design fluido e che sia facilmente usufruibile da mobile.
    Per i colori invece nel sito vi è un predominante utilizzo del classico testo nero su sfondo bianco ma vi sono anche molte sezioni con testo bianco su sfondo verde, come nel menu, che risultano facilmente leggibili e non eccessivamente contrastanti all' occhio.
    Così facendo siamo riusciti ad ottenere un sito accessibile su ogni dispositivo, grazie anche all' utilizzo di media query che vanno a rendere il design del sito scalabile e accessibile.
    
    \paragraph{Presentazione Generale}: Tutte le pagine presentano il logo del parco in cima alla pagina con la possibilità di cliccarlo per venire reindirizzati alla home, appeno sotto di esso si può trovare il menu. Il menu è strutturato nella forma classica con una visualizzazione orrizontale
                                        di tutte le voci principali, passando poi il cursore sulle voci animali ed informazioni verranno visualizzate ulteriori opzioni, la presentazione del menu cambiera quando utilizzato da mobile(vedere sezione "Differenze mobile vs desktop").
                                        In fondo alla pagina si potra infine trovare il footer che ha il principale compito di contenere la dicitura per il copyright.

    \subsection{Principali differenze mobile vs desktop}
        Le seguenti sezioni delinearanno le differenze più importanti che si possono osservare su mobile rispetto alla versione desktop.

        \paragraph{Menu mobile}: Su mobile il menu viene interamento nascosto eccetto la voce "Home" che resta visualizzata come da desktop, per visualizzare il resto del menu invece bastera cliccare sul burger menu in alto a destra e tutte le voci rimanenti compariranno
                                in un menu a tendina

        \paragraph{Home mobile} : Nella Home la sezione "Prossimi eventi al Parco Faunistico Euganeo" e l'immagine non vengono più visualizzati al fianco ma sotto la sezione "CHI SIAMO", questo garantisce una lettura della pagina più semplice è pulita per l'utente.

        \paragraph{Animali mobile}: Le schede degli animali non verranno più visualizzate a due a due ma con una visualizzazione puramente verticale. 

        \paragraph{Eventi mobile}: La data e il prezzo degli eventi visualizzati non vengono più visualizzati sulla stessa linea ma il prezzo verra visualizzato sotto lo data per garantire una lettura migliore per l'utente.		   
    \pagebreak