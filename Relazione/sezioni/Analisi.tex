\section{Analisi}

    \subsection{Utenza}
        L'utenza del sito non può essere inquadrata in un target specifico poiché destinata al pubblico del Parco Faunistico Euganeo il quale non presenta restrizioni o interesse per gruppi definiti. Ciò si traduce nell'esigenza di un sito dalla facile e immediata comprensione e nella piena ottemperanza dei canoni di accessibilità.

    \subsection{Attori}
        Abbiamo individuato tre tipi di attori:
        \begin{itemize}
            \item \textbf{Utente non registrato:} può visitare tutte le pagine del sito, eccetto la pagina "Acquista" e l'area personale; può registrarsi attraverso un apposito form e diventare un utente registrato;
            \item \textbf{Utente registrato:} è un utente che ha i premessi per effettuare acquisti o prenotare la visita agli eventi, può accedere alla propria area riservata avendo la possibilità di visualizzare i propri acquisti e i messaggi;
            \item \textbf{Admin:} è un utente registrato che gode di più permessi di quest'ultimo. Può eliminare e creare eventi, può inserire e eliminare nuovi animali e visualizzare gli acquisti e i messaggi di tutti gli utenti registrati.
        \end{itemize}
        Nella relazione useremo il termine "utente generico" per indicare un utente registrato che non è amministratore.

    \subsection{Pagine del sito}
        In questa sezione sono presentate le varie pagine del sito, con le caratteristiche e le funzionalità che devono avere.
        \subsubsection{Home}
            La homepage del sito, la prima a cui l'utente accede, deve contenere le 5 W e presenta una breve descrizione del parco.
            Prevede 3 sezioni:
            \begin{enumerate}
                \item il testo semplice con le informazioni generiche (5W+descrizione);
                \item una sezione che mostra il prossimo evento del parco;
                \item una sezione con alcune regole da rispettare nel parco.
            \end{enumerate}
            Se si desidera raggiungere la homepage mentre si sta visitando una pagina interna del sito, è possibile cliccare sulla voce home del menù oppure sul titolo/logo presente nell'header.

        \subsubsection{Animali}
            Questa sezione del sito si divide in due pagine:
            \begin{itemize}
                \item tutti gli animali del parco
                \item i cuccioli
            \end{itemize}
            La pagina desiderata si raggiunge cliccando la voce omonima del menù a tendina che appare sotto la voce principale animali. L'utente generico e l'utente loggato vedono e possono interagire allo stesso modo con questa pagina, ovvero possono consultare le informazioni sugli animali semplicemente selezionando e cliccando sopra il nome comune dell'animale.
            La pagina "Tutti gli animali" prevede la visualizzazione di tutti gli animali (nome comune, nome scientifico foto e descrizione) con la possibilità di scorrere tra tutte le voci, oppure fare una ricerca scegliendo di scrivere in un campo editing il nome dell'animale cercato o di una specie in particolare o scegliere da un menù a tendina una categoria predefinita di animali. \\
            La pagina "I cuccioli" prevede la visualizzazione dei soli cuccioli (nome proprio ,nome comune, nome scientifico foto e descrizione). L'utente generico e l'utente loggato vedono e possono interagire allo stesso modo con questa pagina, ovvero possono consultare le informazioni sugli animali.

        \subsubsection{Eventi}
            L’utente generico accede alla pagina eventi cliccando la sezione eventi presente nella navbar, la pagina offre all’utente una panoramica di tutti gli eventi che il parco offre ai visitatori, per ogni evento verranno visualizzate le seguenti informazioni obbligatorie:
            \begin{itemize}
                \item Titolo dell’evento;
                \item Data in cui l’evento si svolge;
                \item Prezzo di acquisto del biglietto per l’evento in Euro;
                \item Una descrizione dell’evento.
            \end{itemize}
            La pagina offre inoltre un filtro per visualizzare solo gli eventi che si svolgono nella data scelta dall’utente, per ogni evento vi è inoltre un tasto “PRENOTA ORA” che se usato dall’utente non loggato rimanda alla pagina di login.
            \paragraph{Utente loggato} Nella pagina eventi, l’utente loggato, oltre ad avere a disposizione tutti i casi d’uso dell’utente generico ha a disposizione la possibilità di utilizzare il tasto “PRENOTA ORA” presente per ogni evento, il tasto rimanderà alla pagina acquista con i filtri già preimpostati sull’evento seleziona.

            \subsubsection{Informazioni}
            L’utente generico accede alla pagina info cliccando la sezione info presente nella navbar, vi è inoltre la possibilità di venire indirizzati direttamente alla sezione contatti della pagina cliccando sul link contatti che viene visualizzata passando il cursore sulla sezione info. La pagina offre informazioni e servizi utili all’utente come:
            L’indirizzo del parco, la mappa con le indicazioni per raggiungere il parco importata da Google Maps, nonostante non sia supportata da XHTML 1.0 Strict è stato scelto di tenerla dato che è compatibile con la maggior parte dei browser e fornisce un servizio non indifferente all’utente che visita la pagina.
            \begin{itemize}
                \item Un indirizzo email;
                \item Un numero di telefono;
                \item Un numero di fax;
                \item Una mappa con le indicazioni per raggiungere il parco importata da Google Maps, nonostante non sia supportata da XHTML 1.0 Strict è stato scelto di tenerla dato che è compatibile con la maggior parte dei browser e fornisce un servizio non indifferente all’ utente che visita la pagina;
                \item Un form che l’utente può utilizzare per mandare messaggi al personale del parco.
            \end{itemize}
            Il form è composto dai campi nome, cognome, email e messaggio da completare obbligatoriamente oltre ad un campo telefono opzionale, dopo essere stato inoltrato il form esegue uno script php per inviare il messaggio in una sezione dedicata del database. Il form è utilizzabile anche da utenti non registrati.

        \subsubsection{Acquista}
            L'utente che non ha eseguito l'accesso non possiede le credenziali necessarie per qualunque tipo di prevendita; per tali considerazioni non potrà in alcun modo visualizzare la pagina: per qualunque provenienza (cliccando un evento dalla pagina Eventi o cliccando Acquista nel menu) verrà rediretto alla pagina di Login dopo aver visualizzato un breve messaggio di warning.
            Per l'utente registrato questa pagina prevede due schede:
            \begin{itemize}
                \item \textbf{Biglietti:} la scheda Biglietti può essere visualizzata cliccandovici dalla scheda "Prenotazione Eventi" oppure cliccando la voce "Acquista" nella barra del menu. L'utente può modificare quantità per tipo di biglietti e confermare l'acquisto attraverso un semplice form. All'acquisto sarà visualizzato un messaggio di conferma e il totale speso all'inizio della pagina stessa, se l'inserimento a database ha dato esito negativo vedrà invece un messaggio di errore.
                \item \textbf{Eventi:} la scheda di prenotazione degli eventi è visualizzata all'apertura della pagina "Acquista" oppure cliccando prenota dal dettaglio di un evento futuro della pagina "Eventi" del sito. L'utente può scegliere l'evento desiderato attraverso un menu a tendina, inserire il numero di Partecipanti e in seguito confermare la prenotazione; in seguito alla quale vedrà un messaggio di conferma o errore esattamente come descritto per la scheda Biglietti.
            \end{itemize}
            \subsubsection{Login e registrazione}
            Un utente non registrato ha la possibilità di registrarsi cliccando il link presente nella pagina Login, raggiungibile cliccando la corrispondente voce del menù. Se si è registrati, si può effettuare il login con le sue credenziali (email e password) e accedere alla proprio area personale.
        \subsubsection{Area Privata}
            L'area privata è diversa a seconda del tipo di utente che ha effettuato. Definiamo "Area personale" l'area privata di un utente generico registrato e loggato; definiamo "Area amministratore" l'area privata di un utente amministratore.
            \paragraph{Area personale} Un utente generico che accede alla sua area privata, come prima pagina vedrà le azioni rapide che può eseguire, come leggere i messaggi, fare acquisti o contattare l'amministratore. Attraverso il "Pannello Gestione" potrà navigare all'interno dell'area privata e accedere alle sottosezioni presenti, ovvero i messaggi, i proprio acquisti (biglietti ed eventi) e visualizzare i propri dati personali.
            \paragraph{Area amministratore} L'area privata dell'utente amministratore presenta due importanti funzionalità: l'inserimento e la cancellazione di un evento o animale. Per l'inserimento è sufficiente cliccare sui pulsanti presenti nelle "Azioni Rapide", mentre nelle sottosezioni dell'area amministrazione "Eventi" e "Animali", l'amministrazione può consultare ed eliminare i dati presenti nel database. Può inoltre consultare gli acquisti e i messaggi di tutti gli utenti registrati.

\pagebreak
