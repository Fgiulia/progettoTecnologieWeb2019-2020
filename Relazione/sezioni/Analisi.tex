\section{Analisi}

    \subsection{Utenza}
        L'utenza del sito non può essere inquadrata in un target specifico poichè destinata al pubblico del parco faunistico Euganeo Lungo Piovego il quale non presenta restrizioni o interesse per gruppi definiti. Ciò si traduce nell'esigenza di un sito dalla facile e immediata comprensione e nella piena ottemperanza dei canoni di accessibilità.

    \subsection{Attori}
        Abbiamo individuato tre tipi di attori:
        \begin{itemize}
            \item \textbf{Utente non registrato:} può visitare tutte le pagine del sito, eccetto la pagina "Acquista" e l'area personale; può registrarsi attraverso un apposito form e diventare un utente registrato;
            \item \textbf{Utente registrato:} è un utente che ha i premessi per effettuare acquisti o prenotare la visita agli eventi, può accedere alla propria area riservata avendo la possibilità di visualizzare i propri acquisti e i messaggi;
            \item \textbf{Admin:} è un utente registrato che gode di più permessi di quest'ultimo. Può eliminare e creare eventi, può inserire e eliminare nuovi animali e visualizzare gli acquisti e i messaggi di tutti gli utenti registrati.
        \end{itemize}
    
    \subsection{Pagine del sito}
        \subsubsection{Home}
            La homepage del sito, la prima a cui l'utente accede, deve contenere le 5 W e presenta una breve descrizione del parco.
            Prevede 3 sezioni:
            \begin{enumerate}
                \item il testo semplice con le informazioni generiche (5W+descrizione);
                \item una sezione che mostra il prossimo evento del parco;
                \item una sezione con alcune regole da rispettare nel parco.
            \end{enumerate}
            Se si desidera raggiungere la homepage mentre si sta visitando una pagina interna del sito, è possibile cliccare sulla voce home del menù oppure sul titolo/logo presente nell'header.

        \subsubsection{Animali}
            Questa sezione del sito si divide in due pagine:
            \begin{itemize}
                \item tutti gli animali del parco
                \item i cuccioli
            \end{itemize}
            La pagina desiderata si raggiunge cliccando la voce omonima del menù a tendina che appare sotto la voce principale animali. L'utente generico e l'utente loggato vedono e possono interagire allo stesso modo con questa pagina, ovvero possono consultare le informazioni sugli animali semplicemente selezionando e cliccando sopra il nome comune dell'animale.
            La pagina tutti gli animali prevede la visualizzazione di tutti gli animali (nome comune, nome scientifico foto e descrizione) con la possibilità di scorrere tra tutte le voci, oppure fare una ricerca scegliendo di scrivere in un campo editing il nome dell'animale cercato o di una specie in particolare o scegliere da un menù a tendina una categoria predefinita di animali. \\
            La pagina i cuccioli prevede la visualizzazione dei soli cuccioli (nome proprio ,nome comune, nome scientifico foto e descrizione). L'utente generico e l'utente loggato vedono e possono interagire allo stesso modo con questa pagina, ovvero possono consultare le informazioni sugli animali.

        \subsubsection{Eventi}
        \subsubsection{Informazioni}
        \subsubsection{Acquista}
        \subsubsection{Login e registrazione}
            Un utente non registrato ha la possibilità di registrarsi cliccando il link presente nella pagina Login, raggiungibile cliccando la corrispondente voce del menù. Se si è registrati, si può effettuare il login con le sue credenziali (email e password) e accedere alla proprio area personale.
        \subsubsection{Area Privata}
            L'area privata è diversa a seconda del tipo di utente che ha effettuato. Definiamo "Area personale" l'area privata di un utente generico registrato e loggato; definiamo "Area amministratore" l'area privata di un utente amministratore.
            \paragraph{Area personale} Un utente generico che accede alla sua area privata, come prima pagina vedrà le azioni rapide che può eseguire, come leggere i messaggi, fare acquisti o contattare l'amministratore. Attraverso il "Pannello Gestione" potrà navigare all'interno dell'area privata e accedere alle sottosezioni presenti, ovvero i messaggi, i proprio acquisti (biglietti ed eventi) e visualizzare i propri dati personali.
            \paragraph{Area amministratore} L'area privata dell'utente amministratore presenta due importanti funzionalità: l'inserimento e la cancellazione di un evento o animale. Per l'inserimento è sufficiente cliccare sui pulsanti presenti nelle "Azioni Rapide", mentre nelle sottosezioni dell'area amministrazione "Eventi" e "Animali", l'amministrazione può consultare ed eliminare i dati presenti nel database. Può inoltre consultare gli acquisti e i messaggi di tutti gli utenti registrati.

\pagebreak