\subsection{Incontro del 03/12/2019}
    Durata: 15.30-18.00

    Primo incontro: esposizione generale sul sito (tematica, utenti) e condivisione della proprio esperienza sullo sviluppo.
    Sono stati affrontati i seguenti punti:
    
    \begin{itemize}
        \item \textbf{nome gruppo:} "Le nutrie"; 
        \item \textbf{nome sito:} "Parco faunistico euganeo"; 
        \item \textbf{strumenti da utilizzare:}
            \begin{itemize}
                \item \textbf{GitHub:} per il versionamento; 
                \item \textbf{Visual Studio Code:} editor per la scrittura del codice; 
                \item \textbf{Draw.io:} usato per UML e bozza grafica del sito;
                \item \textbf{Telegram:} per le comunicazioni interne al gruppo;
            \end{itemize}
    \end{itemize}

    In questo primo incontro è stata fatta un'analisi preliminare del sito e delle sue funzionalità, come pagine del sito,le interazioni che un utente può avere con esso, il metodo di sviluppo, ecc.
    Un'analisi più approfondita verrà fatta individualmente da ogni membro, secondo la seguente suddivisione:
    \begin{itemize}
        \item Giacomo: Area privata
        \item Riccardo: Acquista
        \item Giulia: Home e Animali
        \item Erik: Eventi e Info
    \end{itemize}

\subsection{Incontro del 05/12/2019}
    Durata: 15.30-17.00

    Esposizione analisi delle pagine del sito, discussione generale sulle funzionalità e approvazione delle bozze grafiche.
    Si è scelto di usare pagine HTML pure e di creare i vari elementi servendosi si script PHP, prendendo i dati da un Database.

\subsection{Incontro del 10/12/2019}
    Durata: 15.00-18.00

    Progettazione database, discussione di problemi sorti durante l'analisi delle funzionalità del sito. Compilazione di un documento excel con i macrotask da svolgere.
    Prima fase di progettazione del sito, con scelta del layout grafico del sito e dei colori da utilizzare.
    Scelta dell'organizzazione dei file inerenti al sito web all'interno della repository, che verranno divisi in: HTML, JavaScript e PHP.

    Suddivisione del lavoro e inizio sviluppo del sito.

\subsection{Incontro del 20/12/2019}
    Durata: 15.00-16.30
    Verifica dell'avanzamento del sito web: analisi del corretto sviluppo, discussione e risoluzioni dei problemi e/o dubbi riscontrati durante questa prima fase di sviluppo.
    Pianificazione del lavoro fino alla data del prossimo incontro, fissata per il giorno Martedì 14 Gennaio 2020.

\subsection{Incontro del 20/01/2020}
    Durata: 15.30-18.30
    Verifica dell'effettivo stato di avanzamento del sito. Analisi delle parti e delle funzionalità mancanti o incomplete.
    Primo test generale del corretto funzionamento dei vari form e delle pagine del sito.
    Test visualizzazione su vari browser e dispositivi diversi.

\subsection{Incontro del 23/01/2020}
    Durata: 15.30-17.00
    Caricamento del sito sul server del dipartimento, con sistemazione di puntamenti e connessione al DB. Creazione e popolamento database sul server.
    Verifica del corretto funzionamento del sito in tutte le sue parti. Stesura di una lista di errori o imperfezioni riscontrati.




