\subsection{Incontro del 03/12/19}
    Durata: 15.30-18.00

    Primo incontro: esposizione generale sul sito (tematica, utenti) e condivisione della proprio esperienza sullo sviluppo.
    Sono stati affrontati i seguenti punti:
    
    \begin{itemize}
        \item \textbf{nome gruppo:} "Le nutrie"; 
        \item \textbf{nome sito:} "Parco faunistico euganeo"; 
        \item \textbf{strumenti da utilizzare:}
            \begin{itemize}
                \item \textbf{GitHub:} per il versionamento; 
                \item \textbf{Visual Studio Code:} editor per la scrittura del codice; 
                \item \textbf{Draw.io:} usato per UML e bozza grafica del sito;
                \item \textbf{Telegram:} per le comunicazioni interne al gruppo;
            \end{itemize}
    \end{itemize}

    In questo primo incontro è stata fatta un'analisi preliminare del sito e delle sue funzionalità, come pagine del sito,le
    le interazioni che un utente può avere con esso, il metodo di sviluppo, ecc.
    Un'analisi più approfondita verrà fatta individualmente da ogni membro, secondo la seguente suddivisione:
    \begin{itemize}
        \item Giacomo: Area privata
        \item Riccardo: Acquista
        \item Giulia: Home e Animali
        \item Erik: Eventi e Info
    \end{itemize}

\subsection{Incontro del 05/12/19}